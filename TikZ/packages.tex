%%%%%%%%% Dokumentenklasse
\documentclass[
	crop,
	tikz,
	]{standalone}

\usepackage{siunitx}										% Saubere Darestellung von SI Einheiten
\sisetup{
%	locale = DE,												% Deutsche Norm, Kommas werden z.B. erkannt
	per-mode=fraction,										% Output a/b as \frac{a}{b} - in der Einheit
	quotient-mode=fraction,								% Output a/b as \frac{a}{b} - im Quotienten
	fraction-function=\tfrac,
	range-phrase = {\text{~bis~}},
	sticky-per = true,											% \per bleibt bestehen für mehr als eine Einheit
	separate-uncertainty									% Standardabweichung
}

\usepackage[
	version=4,
	]{mhchem}										% chemische Gleichungen/Summenformeln

%%%%%%%%% Grafiken und Farben
\usepackage{graphicx}									% viele grafische Befehle, z.B. \scalebox
\usepackage{xcolor, colortbl}							% Farben und Farbpaletten
\usepackage[export]{adjustbox}						% Positionieren von Grafiken, left, right, center
\usepackage{float}										% Floating Bilder, H-Befehl
\usepackage{mwe} 										% Fuer Abbildungsverzeichnis
\usepackage{graphics}										% accommodate all needs for inclusion of graphics
\usepackage{subfigure}									%support for the manipulation and reference of small or ‘sub’ figures and tables within a single figure or table environment

%%%%%%%%% Diagramme
\usepackage{tikz,											% Grafik-Paket
	stackengine}												% versatile way to stack objects vertically in a variety of customizable ways
\usepackage{pgfplots}									% Plots
\pgfplotsset{compat=1.14}								% Soll man wohl machen


%%%%%%%%% Seitenformatierung
\usepackage{adjustbox}								% Elemente skalieren \scalebox
	
%%%%%%%%% Schriftart
\usepackage{mathptmx}								% Times New Roman

\usetikzlibrary{arrows,
	patterns,											% Patterns anstatt Farben ermöglichen
	positioning,
	}

\definecolor{rosso}{RGB}{220,57,18}
\definecolor{giallo}{RGB}{255,153,0}
\definecolor{blu}{RGB}{102,140,217}
\definecolor{verde}{RGB}{16,150,24}
\definecolor{viola}{RGB}{153,0,153}

\makeatletter

\tikzstyle{chart}=[
    legend label/.style={font={\scriptsize},anchor=west,align=left},
    legend box/.style={rectangle, draw, minimum size=5pt},
    axis/.style={black,semithick,->},
    axis label/.style={anchor=east,font={\tiny}},
]

\tikzstyle{bar chart}=[
    chart,
    bar width/.code={
        \pgfmathparse{##1/2}
        \global\let\bar@w\pgfmathresult
    },
    bar/.style={very thick, draw=white},
    bar label/.style={font={\bf\small},anchor=north},
    bar value/.style={font={\footnotesize}},
    bar width=.75,
]

\tikzstyle{pie chart}=[
    chart,
    slice/.style={line cap=round, line join=round, very thick,draw=white},
    pie title/.style={font={\bf}},
    slice type/.style 2 args={
        ##1/.style={fill=##2},
        values of ##1/.style={}
    }
]

\pgfdeclarelayer{background}
\pgfdeclarelayer{foreground}
\pgfsetlayers{background,main,foreground}


\newcommand{\pie}[3][]{
    \begin{scope}[#1]
    \pgfmathsetmacro{\curA}{90}
    \pgfmathsetmacro{\r}{1}
    \def\c{(0,0)}
    \node[pie title] at (90:1.3) {#2};
    \foreach \v/\s in{#3}{
        \pgfmathsetmacro{\deltaA}{\v/100*360}
        \pgfmathsetmacro{\nextA}{\curA + \deltaA}
        \pgfmathsetmacro{\midA}{(\curA+\nextA)/2}

        \path[slice,\s] \c
            -- +(\curA:\r)
            arc (\curA:\nextA:\r)
            -- cycle;
        \pgfmathsetmacro{\d}{max((\deltaA * -(.5/50) + 1) , .5)}

        \begin{pgfonlayer}{foreground}
        \path \c -- node[pos=\d,pie values,values of \s]{$\v\%$} +(\midA:\r);
        \end{pgfonlayer}

        \global\let\curA\nextA
    }
    \end{scope}
}

\newcommand{\legend}[2][]{
    \begin{scope}[#1]
    \path
        \foreach \n/\s in {#2}
            {
                  ++(0,-10pt) node[\s,legend box] {} +(5pt,0) node[legend label] {\n}
            }
    ;
    \end{scope}
}

\tikzset{
    %Define standard arrow tip
    >=stealth',
    %Define style for boxes
    punkt/.style={
		rectangle,
		rounded corners,
		draw=black, very thick,
		text width=6.5em,
		minimum height=2em,
		text centered},
	kreis/.style={
		circle,
		draw=black, very thick,
		text width=2.0em,
		minimum height=0.5em,
		text centered},
    % Define arrow style
    pil/.style={
           ->,
           thick,
           shorten <=2pt,
           shorten >=2pt,}
}