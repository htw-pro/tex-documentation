%%%%%%%%% Dokumentenklasse
\documentclass[
	crop,
	tikz,
	]{standalone}

\usepackage{siunitx}										% Saubere Darestellung von SI Einheiten
\sisetup{
%	locale = DE,												% Deutsche Norm, Kommas werden z.B. erkannt
	per-mode=fraction,										% Output a/b as \frac{a}{b} - in der Einheit
	quotient-mode=fraction,								% Output a/b as \frac{a}{b} - im Quotienten
	fraction-function=\tfrac,
	range-phrase = {\text{~bis~}},
	sticky-per = true,											% \per bleibt bestehen für mehr als eine Einheit
	separate-uncertainty									% Standardabweichung
}

\usepackage[
	version=4,
	]{mhchem}										% chemische Gleichungen/Summenformeln

%%%%%%%%% Grafiken und Farben
\usepackage{graphicx}									% viele grafische Befehle, z.B. \scalebox
\usepackage{xcolor, colortbl}							% Farben und Farbpaletten
\usepackage[export]{adjustbox}						% Positionieren von Grafiken, left, right, center
\usepackage{float}										% Floating Bilder, H-Befehl
\usepackage{mwe} 										% Fuer Abbildungsverzeichnis
\usepackage{graphics}										% accommodate all needs for inclusion of graphics
\usepackage{subfigure}									%support for the manipulation and reference of small or ‘sub’ figures and tables within a single figure or table environment

%%%%%%%%% Diagramme
\usepackage{tikz,											% Grafik-Paket
	stackengine}												% versatile way to stack objects vertically in a variety of customizable ways
\usepackage{pgfplots}									% Plots
\pgfplotsset{compat=1.14}								% Soll man wohl machen


%%%%%%%%% Seitenformatierung
\usepackage{adjustbox}								% Elemente skalieren \scalebox
	
%%%%%%%%% Schriftart
\usepackage{mathptmx}								% Times New Roman

\input{tikz_macros.tex}