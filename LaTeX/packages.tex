%%%%%%%%% Dokumentenklasse
%\documentclass[
%	fontsize=12pt,
%	paper=a4,
%	oneside,
%	leqno
%	]{scrartcl}

\documentclass[
	conference,
	paper=a4,
	]{IEEEtran}

%%%%%%%%% Literatur
\usepackage[
	backend = biber,
	]{biblatex}

%%%%%%%%% Sprachpackages
\usepackage[T1]{fontenc}								% Textzeichen für westeuropäische Sprachen
\usepackage[utf8]{inputenc}							% Umlaute
\usepackage[ngerman]{babel}						% Automatisch erzeugte Texte werden auf Deutsch ausgegeben
\usepackage{csquotes}
\usepackage{textcomp}									% Einige Symbole

%%%%%%%%% Mathe

\usepackage{
	amsmath,														% Mathematische Formartierung
	amssymb,													% Mathematische Symbole
	amsfonts,														% Mathematische Schriftarten
	}


%%%%%%%%% Grafiken und Farben
\usepackage{graphicx}									% viele grafische Befehle, z.B. \scalebox
\usepackage{xcolor, colortbl}							% Farben und Farbpaletten
\usepackage[export]{adjustbox}						% Positionieren von Grafiken, left, right, center
\usepackage{float}										% Floating Bilder, H-Befehl
\usepackage{mwe} 										% Fuer Abbildungsverzeichnis
\usepackage[center]{caption} 						% Zentriert Bildunterschriften
\usepackage[labelfont={
		color=black,
		sf,
		it
		},
	font={
	color=black,
	footnotesize,
	it
	},
	labelsep=space
	]{caption}													% customise captions
\usepackage{subfigure}									%support for the manipulation and reference of small or ‘sub’ figures and tables within a single figure or table environment
\usepackage{fancybox}									% kann Boxen (Rahmen) um Elemente legen


%%%%%%%%% Seitenformatierung
\usepackage[
	headsepline,												% Vertikale Linie unterm Header
	plainheadsepline,
	automark													% Section im Header
	]{scrlayer-scrpage}									% Paket zur Manipulation der Kopf- und Fußzeilen
\renewcommand{\headfont}{}						% Voreingestellte Schriftart bearbeiten, z.B. möglich \itshape für ein feines Kursiv
\usepackage[onehalfspacing]{setspace}			% Setzt Zeilenabstand auf 1.5

	
%%%%%%%%% Schritart
%\usepackage{mathptmx}								% Times New Roman


%%%%%%%%% Tabellenumgebung
\usepackage{booktabs}									% enhances the quality of tables
\usepackage{array}										% extends the options for column formats
\usepackage{multirow}									% mehrzeilige Tabellenzellen
\usepackage{tabularx}									% Tabellenumgebung, praktisch für Textweite-Tabellen, Standard Column-Type: X
\newcolumntype{R}{>{\raggedleft\arraybackslash}X}	% Neuer rechtsbündiger Column-Type


%%%%%%%%% Datumsformat \today
\usepackage[ngerman, num]{isodate}				% \today im DD. MM. YYYY Format
\daymonthsepgerman{}{}								%
\monthyearsepgerman{}{}								% entfernt Leerzeichen nach den Punkten


%%%%%%%%% Kopf- und Fußzeile
%\ofoot*{\today}
\cfoot{Seite \pagemark}				% Mittleres Foot-Element
\ohead{ }										% Obere Seitenzahl disablen

%%%%%%%%% Abkürzungsverzeichnis:
\usepackage[													% Paket für Glossaries und Acronym-Glossaries
	acronym,
	automake,
	nopostdot
	]{glossaries}

	
%\usepackage{tocloft}
%\setlength\cftparskip{0pt}
%\setlength\cftbeforesecskip{0pt}
%\setlength\cftaftertoctitleskip{1pt}