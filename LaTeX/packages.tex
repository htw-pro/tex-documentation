%%%%%%%%% Dokumentenklasse
%\documentclass[
%	fontsize=12pt,
%	paper=a4,
%	oneside,
%	leqno
%	]{scrartcl}

\documentclass[
	conference,
	]{IEEEtran}
	
%%%%%%%%% Formatierung

\makeatletter
\usepackage{microtype}									% Verbesserte Formatierung; Sollte immer geladen werden
\g@addto@macro\@verbatim{\microtypesetup{activate=false}}
\makeatother

%%%%%%%%% Literatur
\usepackage[
	backend = biber,
	]{biblatex}

%%%%%%%%% Sprachpackages
\usepackage[T1]{fontenc}								% Textzeichen für westeuropäische Sprachen
\usepackage[utf8]{inputenc}							% Umlaute
\usepackage[ngerman]{babel}						% Automatisch erzeugte Texte werden auf Deutsch ausgegeben
\usepackage[ngerman]{datetime2}						% Datumsformate
\usepackage{csquotes}
\usepackage{textcomp}									% Einige Symbole
\usepackage[official]{eurosym}							% €-Symbol

%%%%%%%%% Mathe & Naturwissenschaften
\usepackage{
	amsmath,														% Mathematische Formartierung
	amssymb,													% Mathematische Symbole
	amsfonts,														% Mathematische Schriftarten
	}
\usepackage{siunitx}										% Saubere Darestellung von SI Einheiten
\sisetup{
%	locale = DE,												% Deutsche Norm, Kommas werden z.B. erkannt
	per-mode=fraction,										% Output a/b as \frac{a}{b} - in der Einheit
	quotient-mode=fraction,								% Output a/b as \frac{a}{b} - im Quotienten
	fraction-function=\tfrac,
	range-phrase = {\text{~bis~}},
	sticky-per = true,											% \per bleibt bestehen für mehr als eine Einheit
	separate-uncertainty									% Standardabweichung
}

\usepackage{mhchem}										% chemische Gleichungen/Summenformeln

%%%%%%%%% Grafiken und Farben
\usepackage{graphicx}									% viele grafische Befehle, z.B. \scalebox
\usepackage{xcolor, colortbl}							% Farben und Farbpaletten
\usepackage[export]{adjustbox}						% Positionieren von Grafiken, left, right, center
\usepackage{float}										% Floating Bilder, H-Befehl
\usepackage{mwe} 										% Fuer Abbildungsverzeichnis
%\usepackage[
%	center												% Zentriert Bildunterschriften
%	]{caption}
%\captionsetup[table]{skip=5pt}
%\usepackage[labelfont={
%		color=black,
%		sf,
%		%it
%		},
%	font={
%	color=black,
%	footnotesize,
%	it
%	},
%	labelsep=space
%	]{caption}													% customise captions
\usepackage{subfigure}									%support for the manipulation and reference of small or ‘sub’ figures and tables within a single figure or table environment

%%%%%%%%% Diagramme
%\usepackage{tikz,											% Grafik-Paket
%	stackengine}												% versatile way to stack objects vertically in a variety of customizable ways
\usepackage{pgfplots}									% Plots
\pgfplotsset{compat=1.14}								% Soll man wohl machen
\usetikzlibrary{patterns}									% Patterns anstatt Farben ermöglichen

\usepackage{environ}										% Textwidth von tikzpictures
\makeatletter
\newsavebox{\measure@tikzpicture}
\NewEnviron{scaletikzpicturetowidth}[1]{%
  \def\tikz@width{#1}%
  \def\tikzscale{1}\begin{lrbox}{\measure@tikzpicture}%
  \BODY
  \end{lrbox}%
  \pgfmathparse{#1/\wd\measure@tikzpicture}%
  \edef\tikzscale{\pgfmathresult}%
  \BODY
}
\makeatother

%\usetikzlibrary{arrows,
	patterns,											% Patterns anstatt Farben ermöglichen
	positioning,
	}

\definecolor{rosso}{RGB}{220,57,18}
\definecolor{giallo}{RGB}{255,153,0}
\definecolor{blu}{RGB}{102,140,217}
\definecolor{verde}{RGB}{16,150,24}
\definecolor{viola}{RGB}{153,0,153}

\makeatletter

\tikzstyle{chart}=[
    legend label/.style={font={\scriptsize},anchor=west,align=left},
    legend box/.style={rectangle, draw, minimum size=5pt},
    axis/.style={black,semithick,->},
    axis label/.style={anchor=east,font={\tiny}},
]

\tikzstyle{bar chart}=[
    chart,
    bar width/.code={
        \pgfmathparse{##1/2}
        \global\let\bar@w\pgfmathresult
    },
    bar/.style={very thick, draw=white},
    bar label/.style={font={\bf\small},anchor=north},
    bar value/.style={font={\footnotesize}},
    bar width=.75,
]

\tikzstyle{pie chart}=[
    chart,
    slice/.style={line cap=round, line join=round, very thick,draw=white},
    pie title/.style={font={\bf}},
    slice type/.style 2 args={
        ##1/.style={fill=##2},
        values of ##1/.style={}
    }
]

\pgfdeclarelayer{background}
\pgfdeclarelayer{foreground}
\pgfsetlayers{background,main,foreground}


\newcommand{\pie}[3][]{
    \begin{scope}[#1]
    \pgfmathsetmacro{\curA}{90}
    \pgfmathsetmacro{\r}{1}
    \def\c{(0,0)}
    \node[pie title] at (90:1.3) {#2};
    \foreach \v/\s in{#3}{
        \pgfmathsetmacro{\deltaA}{\v/100*360}
        \pgfmathsetmacro{\nextA}{\curA + \deltaA}
        \pgfmathsetmacro{\midA}{(\curA+\nextA)/2}

        \path[slice,\s] \c
            -- +(\curA:\r)
            arc (\curA:\nextA:\r)
            -- cycle;
        \pgfmathsetmacro{\d}{max((\deltaA * -(.5/50) + 1) , .5)}

        \begin{pgfonlayer}{foreground}
        \path \c -- node[pos=\d,pie values,values of \s]{$\v\%$} +(\midA:\r);
        \end{pgfonlayer}

        \global\let\curA\nextA
    }
    \end{scope}
}

\newcommand{\legend}[2][]{
    \begin{scope}[#1]
    \path
        \foreach \n/\s in {#2}
            {
                  ++(0,-10pt) node[\s,legend box] {} +(5pt,0) node[legend label] {\n}
            }
    ;
    \end{scope}
}

\tikzset{
    %Define standard arrow tip
    >=stealth',
    %Define style for boxes
    punkt/.style={
		rectangle,
		rounded corners,
		draw=black, very thick,
		text width=6.5em,
		minimum height=2em,
		text centered},
	kreis/.style={
		circle,
		draw=black, very thick,
		text width=2.0em,
		minimum height=0.5em,
		text centered},
    % Define arrow style
    pil/.style={
           ->,
           thick,
           shorten <=2pt,
           shorten >=2pt,}
}

%%%%%%%%% Seitenformatierung
\usepackage[
	headsepline,												% Vertikale Linie unterm Header
	plainheadsepline,
	automark,													% Section im Header
	singlespacing=true,
	]{scrlayer-scrpage}									% Paket zur Manipulation der Kopf- und Fußzeilen
\renewcommand{\headfont}{}						% Voreingestellte Schriftart bearbeiten, z.B. möglich \itshape für ein feines Kursiv
\usepackage[
	onehalfspacing,
	]{setspace}												% Setzt Zeilenabstand auf 1.5
\usepackage{adjustbox}								% Elemente skalieren \scalebox
	
%%%%%%%%% Schriftart
%\usepackage{mathptmx}								% Times New Roman


%%%%%%%%% Tabellenumgebung
\usepackage{booktabs}									% enhances the quality of tables
\usepackage{array}										% extends the options for column formats
\usepackage{multirow}									% mehrzeilige Tabellenzellen
%\usepackage{tabularx}									% Tabellenumgebung, praktisch für Textweite-Tabellen, Standard Column-Type: X
\usepackage{tabu}											% Moderneres tabularx
\newcolumntype{R}{>{\raggedleft\arraybackslash}X}	% Neuer rechtsbündiger Column-Type

%%%%%%%%% Datumsformat \today
\usepackage[ngerman, num]{isodate}				% \today im DD. MM. YYYY Format
\daymonthsepgerman{}{}								%
\monthyearsepgerman{}{}								% entfernt Leerzeichen nach den Punkten


%%%%%%%%% Kopf- und Fußzeile
%\ofoot*{\today}
\cfoot{Seite \pagemark}				% Mittleres Foot-Element
\ohead{ }										% Obere Seitenzahl disablen
	
%\usepackage{tocloft}
%\setlength\cftparskip{0pt}
%\setlength\cftbeforesecskip{0pt}
%\setlength\cftaftertoctitleskip{1pt}

\newenvironment{conditions}						% New environment - Für saubere Darstellung gegebener Variablen
	{\par\vspace{\abovedisplayskip}\noindent\begin{tabular}{>{$}l<{$} @{${}={}$} l}}
	{\end{tabular}\par\vspace{\belowdisplayskip}}

% Roman Numbers in Text
\makeatletter
\newcommand*{\rom}[1]{\expandafter\@slowromancap\romannumeral #1@}
\makeatother

\usepackage[													% Muss das letzte Paket sein was lädt, außer glossaries
%	hidelinks,														% ohne Boxen um den Link im pdf
	]{hyperref}													% Links im pdf

\hypersetup{
    colorlinks,
    linkcolor={red!50!black},
    citecolor={blue!50!black},
    urlcolor={blue!80!black}
}

%%%%%%%%% Abkürzungsverzeichnis:
\usepackage[													% Paket für Glossaries und Acronym-Glossaries
	acronym,														% Muss das letzte Paket sein was lädt
	automake,
	nopostdot,
	toc,
	nomain,
	shortcuts,
	]{glossaries}

\renewcommand*{\arraystretch}{0.6}  % sets the line indent in glossaries