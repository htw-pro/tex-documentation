% test

\section{Techno-ökonomische Analyse}\label{chap:chap_tececon}

In diesem Kapitel werden die Verfahren zur Biogasaufbereitung und Methanisierung (s. Kap. \ref{chap:stateofart}) näher auf ihre Effizienz und Kosten untersucht. Abschließend soll ermittelt werden, wie hoch der Preis für eine Kilowattstunde Biomethan aus Biogas ist. Danach soll außerdem bewertet werden, ob die Biomethanherstellung einen validen Flexibilisierungspfad für Bestandsanlagen darstellt. 

\subsection{Analyse der konventionellen Aufbereitungsverfahren}
Zunächst werden die konventionellen Aufbereitungsverfahren untersucht. Da die Biogasaufbereitung in Biogasbestandsanlagen betrachtet wird, werden Gasreinigungsverfahren wie Entschwefelung vorausgesetzt und nicht weiter gewichtet.  \smallskip


\subsubsection{Druckwasserwäsche}
Die \glsfirst{DWW} ist derzeit das am meisten genutzte Biogasaufbereitungsverfahren. Da das Waschmittel Wasser ist, werden keine teuren oder umweltschädlichen Chemikalien benötigt. Das Produktgas erreicht eine \ce{CH4} Reinheit von \SIrange{95}{99}{\percent}. Die Hauptvorteile dieser Technologie sind, dass sie ausgereift ist, das Anlagendesign recht einfach ist und sie sehr günstig ist. Der \ce{CH4}-Schlupf ist mit ungefähr \SI{2}{\percent} akzeptabel. In einzelnen Anlagen ist ein Schlupf von \SIrange{8}{10}{\percent} aufgetreten, was auf ein schlecht optimiertes System zurückzuführen ist. Ein Nachteil ist, dass die Energieeffizienz, verglichen mit anderen Prozessen, mit \SI{0,3}{\kwh\per\normvol} Biogas recht gering ist. Dabei fällt ein Großteil des Energieverbrauchs bei der Gasverdichtung an. Je nach Quelle ist die \gls{DWW} aber auch schon ab \SI{0,2}{\kwh\per\normvol} möglich. \parencite{Emp18} \parencite{Coll17}\smallskip

Die Investitionskosten betragen, je nach Größe der Anlage, \SIrange{2500}{5500}{\sieuro\per\normvolh} für eine Anlagengröße von \SIrange{100}{500}{\normvol\per\hour} und \SIrange{1800}{2000}{\sieuro\per\normvolh} für Anlagen größer als \SI{1000}{\normvolh}. Letztendlich liegen die Kosten für das produzierte Biomethan zwischen \SIrange{0,13}{0,16}{\sieuro\per\normvol}. \parencite{Emp18} \parencite{KGKK2019} \parencite{Struk20}\smallskip

Wegen seiner niedrigen Kosten für die Gasaufbereitung wird die \gls{DWW} als eine vielversprechende Technologie für die Methaneinspeisung einzelner Biogasanlagen eingeschätzt.\smallskip


\subsubsection{Polyglykolwäsche}
Obwohl bei der Polyglykolwäsche weniger Lösungsmittel benötigt wird als bei der \gls{DWW}, wird für die Regeneration mehr Energie benötigt, da das Waschmittel erhitzt werden muss. Der Strombedarf liegt bei \SIrange{0,24}{0,33}{\kwh\per\normvol} Biogas. Zum Teil wird in der Literatur außerdem angegeben, dass zusätzlich eine Wärmezufuhr von \SIrange{0,1}{0,15}{\kwh\per\normvol} Biogas notwendig ist. Andere Quellen geben an, dass die Wärmezufuhr aus der Schwachgasbehandlung für die Lösungsmittelregeneration ausreichend ist. \parencite{Emp18}\smallskip

Durch organische Wäsche wird dafür allerdings auch ein hoher Methangehalt von \SIrange{96}{98}{\percent} erreicht und die Methanverluste liegen in der Regel unter \SI{2}{\percent}. Auch bei Polyglykolwäsche sind die Investitionskosten stark abhängig von der Anlagengröße. Bei einem Rohgasstrom von weniger als \SI{500}{\normvolh} Biogas sind die Kosten mit \SIrange{3000}{4500}{\sieuro\per\normvolh} sehr hoch. Bei einer Anlage größer \SI{1000}{\normvolh} liegen die Kosten nur noch bei etwa \SI{2000}{\sieuro\per\normvolh}. Die Kosten für die Gasaufbereitung liegen umgerechnet bei \SIrange{0,09}{0,26}{\sieuro\per\normvol}. \parencite{FNR14}\smallskip

Insgesamt ist für die Flexibilisierung von Biogasanlagen auf Grundlage Polyglykolwäsche mit Kosten von über \SI{0,20}{\sieuro\per\normvol} zu rechnen. Das organische Lösemittel Polyethylenglykol wird zudem als wassergefährdend eingestuft. Nach umweltfreundlichen Alternativen wird daher bereits gesucht. Derzeit werden dahingehend stark eutektischen Lösungsmitteln (DES) untersucht, die ähnliche Eigenschaften und Ergebnisse liefern sollen wie Polyglykol. Bis diese neuen Lösungsmittel in der Praxis eingesetzt werden können bedarf es allerdings noch längerer Forschung. \parencite{FNR14} \parencite{Struk20}


\subsubsection{Aminwäsche}
Mit Hilfe chemischer Wäsche wird eine noch höhere Methanreinheit von über \SI{98}{\percent} erreicht, während nur \SI{0,06}{\percent} des Methans verloren gehen. Ein großer Nachteil ist allerdings der Gesamtenergieaufwand, der bei etwa \SIrange{0,62}{0,69}{\kwh\per\normvol} liegt und die Betriebskosten stark prägt. Davon werden \SI{0,55}{\kwh\per\normvol} in Form von Wärme zur Regeneration des Waschmittels benötigt. Kleine Aminwäsche-Anlagen mit einer Rohgaskapazität von \SI{500}{\normvolh} kosten etwa \SI{3000}{\sieuro\per\normvolh} Biogas. Für große Anlagen mit einer Kapazität von \SI{1800}{\normvol\per\hour} liegen die Investitionskosten nur noch bei \SI{1500}{\sieuro\per\normvolh}. Der Preis des resultierenden Biomethans ist recht hoch und wird, je nach Quelle, zwischen \SIrange{0,17}{0,35}{\sieuro\per\normvol} angegeben. Zu beachten ist auch, dass das Abgas mit \SI{99,99}{\percent} eine sehr hohe \ce{CO2} Reinheit besitzt, sodass es geeignet dafür ist als Rohstoff verkauft zu werden. \parencite{Emp18} \parencite{BHPT13} \parencite{DSW15} \parencite{KGKK2019} \parencite{Coll17}


\subsubsection{Membrantrennverfahren}
Membrantrennverfahren können modular aufgebaut werden und sind deshalb skalierbar und können an die individuelle Anforderungen der Anlage angepasst werden. Beispielsweise ist es möglich verschiedene Membranen zu kombinieren, sodass auch andere Gasanteile neben \ce{CO2} entfernt werden können. Um eine Methanreinheit von \SI{95}{\percent} zu erreichen wird ein mindestens zweistufiger Prozess benötigt. Die aufzuwendende Energie ist mit \SI{0,20}{\kwh\per\normvol} Rohbiogas dennoch recht niedrig. Allerdings liegt der \ce{CH4}-Gehalt im Abgas hier immer noch bei etwa \SI{7,2}{\percent}, sodass zusätzliche Verfahrensschritte zu empfehlen sind, um die Verluste zu verringern. Nach vier Stufen liegt der Anteil im Abgas nur noch bei etwa \SI{0,8}{\percent} und es wird ein sehr hoher Methananteil von \SI{99,5}{\percent} erreicht. Die Kosten für die Aufbereitung liegen dann bei ca. \SI{0,15}{\sieuro\per\normvol} Biomethan. Weitere Quellen nennen allerdings Preise von bis zu \SI{0,27}{\sieuro\per\normvol}. Die Membranverfahren sind kommerziell verfügbar, allerdings sind die Investitionskosten stark abhängig von der geplanten Größe der Anlage. Kleine Anlagen, unter \SI{200}{\normvolh}, kosten \SIrange{2500}{6000}{\sieuro\per\normvolh} \parencite{BHPT13}. Dagegen müssen ab einer Größe von \SI{1000}{\normvolh} nur noch ca. \SI{2000}{\sieuro\per\normvolh} investiert werden. Zu beachten ist außerdem, dass die Membranen nach \SIrange{5}{10}{\relax} Jahren ausgetauscht werden müssen. \parencite{Emp18} \parencite{BHPT13} \parencite{KGKK2019} \parencite{Miltner2016}


\subsubsection{Druckwechseladsorption}
Bei der \gls{PSA} hat das Produktgas einen \ce{CH4}-Anteil von \SIrange{96}{99}{\percent}. Der Methanverlust im Abgas beträgt zwischen \SIrange{1}{3}{\percent}. Energie wird in erster Linie zum Aufbauen des Betriebsdrucks benötigt. Der nötige Energieaufwand liegt bei etwa \SIrange{0,15}{0,35}{\kWh\normvol}. Das Verfahren ist also durchaus konkurrenzfähig mit der \gls{DWW}. Dagegen sind die Kapitalkosten für kleine Anlagen recht hoch, da das Anlagesystem recht komplex ist. Sie betragen etwa \SI{2800}{\sieuro\per\normvolh} Biomethan bei einem Biogasstrom von \SI{600}{\normvolh} und \SI{2000}{\sieuro\per\normvolh} bei \SI{1000}{\normvolh}. Die Kosten für die Aufbereitung liegen somit \SIrange{0,25}{0,31}{\sieuro\per\normvol} ebenfalls in einem hohem Bereich. Hingegen spricht für den Prozess, dass keine Chemikalien benötigt werden. \parencite{AONC2019} \parencite{BHPT13} \parencite{dena2019} \parencite{DSW15} \parencite{KGKK2019} \parencite{Coll17}


\subsubsection{Kryogene Trennverfahren}
Die kryogenen Trennprozesse sind noch recht neu. Mit Hilfe dieser Technologie, können sehr hohe Methanreinheiten von \SIrange{97}{99}{\percent} erzielt werden. Der Methanschlupf liegt zudem unter \SI{2}{\percent}. Konsistente Angaben zum benötigten Energieaufwand existieren bisher nicht. Die Angabe unterscheiden sich zum Teil beträchtlich und liegen zwischen \SIrange{0,18}{1}{\kwh\normvol}. Trotzdem ist der Konsens, dass sie aufgrund der starken Kühlung des Gases höher als bei den übrigen konventionellen Technologien liegen. Die Kosten für die Gasaufbereitung sind deswegen mit \SIrange{0,44}{0,55}{\sieuro\per\normvol} dementsprechend hoch. Hier ist anzumerken, dass sich der Preis auf den Prozess mit dem geringeren Energieaufwand bezieht. Unter Umständen liegt der reale Preis je nach Anlagenoptimierung etwas höher. Vorteile des Prozesses sind, dass keine Chemikalien verwendet werden und reines \ce{CO2} als Nebenprodukt entsteht. Trotzdem wird die Technologie bisher nur bei wenigen Anlagen eingesetzt, da die Technik eine Vielzahl von Betriebsmitteln benötigt und energieintensiv ist. Die Investitionskosten für kleine Anlagen liegen etwa bei \SI{5600}{\sieuro\per\normvolh}. Der Einsatz kryogener Verfahren in bestehenden Biogasanlagen ist demnach wegen der hohen Investitions- und Betriebskosten nicht anzuraten. \parencite{AONC2019} \parencite{Emp18} \parencite{KGKK2019}


\subsubsection{Hybridverfahren}
Mit der Kombination verschiedener Verfahren können deren Vor- und Nachteile teilweise ausgeglichen werden. Beispielsweise wird bereits in industriellem Maßstab ein Hybridverfahren aus Membranverfahren und kryogener Gasaufbereitung angewandt. An erster Stelle steht hierbei eine Membrantrennung. Das übrige im Abgas enthaltene \ce{CO2} wird anschließend durch kryogene Trennung abgeschieden. So soll eine hohe Methanreinheit erreicht werden ohne den für die Tieftemperaturkühlung üblichen hohen Energieaufwand. Allerdings liegt der Energieaufwand \SIrange{0,35}{0,37}{\sieuro\per\normvol} Biogas immer noch recht hoch. \parencite{dena2019}\smallskip
%Möglichkeit Hybridverfahren: Beispiel Kryo Membran --> 1 Anlage in Deutschland


\subsubsection{Fazit}
Es stellt sich heraus, dass die \gls{DWW}, zumindest für einzelne Biogasanlagen, das geeignetste Aufbereitungsverfahren ist. Sie liefert eine hohe Produktgasqualität und ist sowohl im Betrieb als auch in der Anschaffung preiswert. Zudem ist sie ökologisch unbedenklich.\smallskip

In Tab. \ref{tab:tab_TechComp} sind die Kennzahlen der einzelnen Technologien noch einmal zusammenfassend dargestellt. Die Biogasaufbereitung kostet demnach mit \gls{DWW} \SIrange{0,13}{0,16}{\sieuro\per\normvol} Methan, also \SIrange{1,2}{1,5}{\ct\per\kwh} ausgehend von einem Brennwert von \SI{11,03}{\kwh\per\normvol}. Das Ergebnis deckt sich mit den Angaben mehrerer Firmen, deren Kosten je nach Verfahren und Anlagengröße bei \SIrange{0,7}{2,5}{\ct\per\kwh} liegen. Die Biogasbereitstellungskosten sind abhängig von der Art des Substrats sowie der Anlagengröße. Dabei liegen sie bei \SIrange{5,0}{6,5}{\ct\per\kwh}. Nach diesen Annahmen ergeben sich für Biomethan also Kosten von ungefähr \SIrange{6,2}{8,0}{\ct\per\kwh}. Biomethan aus Abfallverwertung ist dabei mit durchschnittlich \SI{5,88}{\ct\per\kwh} am günstigsten während Biomethan aus Gülle mit im Schnitt \SI{7,78}{\ct\per\kwh} am teuersten ist. Der Preis für Biomethan aus nachwachsenden Rohstoffen liegt mit durchschnittlich \SI{7,06}{\ct\per\kwh} dazwischen. \parencite{FNR14} \parencite{meier14} \parencite{dena19}

% schrecklich große Tabelle

{
\renewcommand{\arraystretch}{1.1}
\begin{table}[H]
	\begin{center}
		\caption{Anforderungen an Gas aus regenerativen Quellen \parencite{FNR14}\parencite{KoBi16}}
		\begin{tabu} to 0.5\textwidth {| X | X | X | X | X | X | X | X | X |}
\hline
Technologie 			& \ce{CH4} Reinheit 				& \ce{CH4} Verlust 		& 
Energieaufwand 			& Inv. Kosten \si{\sieuro\per\kW} 	& Preis \si{\sieuro\per\normvol} & 
Vorteile 				& Nachteile 						& Forschungsbedarf		\\		
\hline
			\multirow{2}{*}{\ce{CH4}} &	L-Gasnetze: \SI{\geq 90}{\Molpercent}  &	brennbare Gaskomponente	\\
					 &	H-Gasnetze: \SI{\geq 95}{\Molpercent}	&	{}							\\ 
\hline

			
		\end{tabu}
		\label{tab:tab_TechComp}
	\end{center}
\end{table}
}


\subsection{Verarbeitung des Kohlenstoffdioxidanteils}

Im vorherigen Abschnitt ist das \ce{CO2} nach der Abscheidung nicht weiter betrachtet worden. Anstatt es als Abgas in die Atmosphäre zu leiten kann es aber auch in \gls{Ptg}-Prozessen weiterverwendet werden. Werden die nötigen Anforderungen an die Reinheit erfüllt, besteht die Möglichkeit es zu vermarkten. Zwar ist diese Option derzeit noch nicht relevant, sie kann sich aber mit steigenden \gls{Ptg} Kapazitäten als gutes Mittel zur Steigerung der eigenen Wirtschaftlichkeit erweisen. \parencite{UmBA19} \smallskip

Um die Produktion einer Anlage zu erhöhen kann das abgeschiedene \ce{CO2} außerdem genutzt werden, um Methan herzustellen, wie in Kap. \ref{chap:SoR} dargestellt. Die dafür geeigneten Technologien sind der Sabatier-Prozess und die biologische Methanisierung, wobei dafür zusätzliches \ce{H2} als Rohstoff benötigt wird. Die In-situ-Verfahren und die direkte Methanisierung werden nicht weiter untersucht, da sie sich noch in der Entwicklungsphase befinden und weiterer Optimierung bedürfen. Das Kapital, welches für die Anschaffung eines Elektrolyseurs und für die Prozesse der separaten Methanisierung benötigt wird, ist nicht unerheblich. Die Investitionskosten für Elektrolyseure liegen derzeit bei \SIrange{500}{1500}{\Eurkw} für alkalische und \SIrange{800}{1800}{\Eurkw} für PEM-Elektrolyseure. Auch bei der Verarbeitung des Kohlenstoffdioxidanteils sind die Kosten stark abhängig von der Größe der Anlage. Hinzu kommen Stromkosten für die Elektrolyse, dessen Wirkungsgrad nur bei ungefähr \SI{80}{\percent} liegt. Außerdem müssen die Kosten für die Methanisierungsanlage an sich mit einbezogen werden. Für katalytische Methanisierung ist \SI{2020}{\relax} mit \SIrange{400}{1250}{\Eurkw} zu rechnen. Für biologische Methanisierung sind Investitionskosten von \SIrange{300}{1250}{\Eurkw} nötig. Am Beispiel der biologischen Methanisierung ergibt sich ein Preis für Methan von \SIrange{17}{29}{\ctkwh} inklusive der Elektrolysekosten. Umgerechnet sind das etwa \SIrange{1,90}{3,20}{\sieuro\per\normvol}. Das ist zehnmal so viel, wie das Biomethan aus der Aufbereitung kosten würde. Ein möglicher Ansatz die Kosten zu reduzieren ist es überschüssigen Strom aus \glspl{FEE} für die Elektrolyse zu verwenden und damit zum einen Regelleistung zu erbringen und zum anderen überschüssige Energie in Form von Methan zu speichern, wie es die Electrochaea in Dänemark bereits tut. Zum derzeitigen Stand und für einzelne Biogasanlagen stellt die Methanisierung des abgeschiedenen \ce{CO2} allerdings keine Option dar. \parencite{Chaea20} \parencite{4.2b17} \parencite{dena2018b}


\subsection{Ausblick}

Biomethan besitzt große energiesystemtechnische Vorteile, da es flexibel einsetzbar ist und eine regenerative Alternative zu fossilen Kraftstoffen darstellt. Die Kosten für das Gas liegen bei etwa \SIrange{6,2}{8,0}{\ctkwh}, von denen \SIrange{1,2}{1,5}{\ct\per\kwh} für die Biogasaufbereitung anfallen. Jedoch fallen für die Aufbereitungssysteme hohe Investitionskosten an, welche insbesondere hoch sind für kleine Anlagen. Es ist von daher wichtig, den Anlagenbetreibern einen stabilen Cashflow zu garantieren, um die Finanzierungskosten möglichst gering zu halten. Für eine Vielzahl von bestehenden Biomethananlagen sind die vermiedenen Netzkosten (s. Kap. \ref{chap:law_vN}) derzeit noch die Voraussetzung für einen wirtschaftlichen Betrieb. Diese sind inzwischen auf zehn Jahre begrenzt, sodass der Planungshorizont begrent ist und somit hohe Finanzierungskosten anfallen. Biomethan wird nach Einschätzung der dena aber, ohne Anpassung von rechtlichen Rahmenbedingungen, durch den hohen Konkurrenzdruck durch fossiles Erdgas wirtschaftlich eher noch weiter an Attraktivität verlieren.  \parencite{dena2018} \smallskip

Nach derzeitigem Stand stehen also hohe Investitionskosten verbunden mit unsicheren Ertragsmöglichkeiten der Flexibilisierung von Biogasanlagen durch Methanisierung entgegen. Prinzipiell ist zu sagen, dass größere Biogasanlagen mit hohen Produktgasströmen eher für Gasaufbereitung geeignet sind als kleine Anlagen, da die Kosten stark abhängig von der Anlagengröße sind. Rein technisch besteht eine Vielzahl von ausgereiften, kommerziell bereits genutzten Verfahren. Insbesondere die \gls{DWW} stellt dabei eine preiswerte, einfache Aufbereitungstechnologie dar. Die zusätzlicher Methanisierung des abgetrennten \ce{CO2} ist bisher ökonomisch nicht zu empfehlen, da hier für geringe Kapazitäten recht hohe, zusätzliche Investitionen getätigt werden müssen. \smallskip

Sobald Bestandsanlagen ihre Förderung durch das \gls{EEG} verlieren sind die meisten Biogasanlagen nicht mehr wirtschaftlich und werden voraussichtlich zurückgebaut. Um die Flexibilisierung als Biomethananlage attraktiver zu gestalten und den \ce{CO2}-Ausstoß mithilfe erneuerbarer Energieträger zu reduzieren, sind deshalb dringend weitere Förderungen und die Anpassung rechtlicher Rahmenbedingungen nötig. Eine Möglichkeiten die Wirtschaftlichkeit zu verbessern, wäre beispielsweise die Verbesserung des Status von Biomethan, gerade im Vergleich zu fossilem Erdgas. Außerdem würden gesetzliche Anreize für den Zusammenschluss mehrerer kleiner Biogasanlagen zum Bau zentraler Aufbereitungsanlagen helfen das Risiko für große Investitionen zu verringern. \parencite{UmBA19}