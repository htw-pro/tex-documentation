% Schema der Biogasherstellung

\begin{figure}[H]
	\begin{center}
	\resizebox{0.5\textwidth}{!}{
	\begin{tikzpicture}[
		node distance=0cm,
		auto,
		]
		% nodes
		\node[
			align=center,] (hydro) {Hydrolyse und Acidogenese\\von Biomasse};
		\node[below=0.5cm of hydro,
			align=center] (fatty) {Flüchtige Fettsäuren};
		\node[below=0.4 of fatty] (dummy) {};
		\node[below=0.1cm of dummy] (dummy2) {};
		\node[punkt,
			below left=of fatty,
			yshift=-0.5cm,
			align=center] (HCO) {\ce{H2}, \ce{CO2}};
		\node[below=0.5cm of HCO,
			align=center] (HydMet) {Hydrogenotrophe\\Methanogenese};
		\node[punkt,
			below right=of fatty,
			yshift=-0.5cm,
			align=center] (Acetat) {Acetat};
		\node[below=0.5cm of Acetat,
			align=center] (AceMet) {Acetoklastische \\Methanogenese};
		\node[kreis,
			above left=of HCO,
			yshift=0.5cm,
			align=center] (H) {\texttt{+}\ce{H2}};
		\node[below=2.5cm of fatty,
			align=center] (CHCO) {\textbf{\ce{CH4} \texttt{+} \ce{CO2}}};
		% arrows
		\draw[pil,->] 	(H)	--	(HCO);
		\draw[pil,->]	(HCO)	--	(CHCO);
		\draw[pil,->]	(Acetat)	--	(CHCO);
		\draw[pil,->]	(HCO)	--	(Acetat) node[midway,
			sloped,
			above,
			%rotate=270,
			] {Homoacetogenese};
		\draw[pil,->] 	(hydro)	--	(fatty);
		\draw[pil,-] 	(fatty)	--	(dummy);
		\draw[pil,->] 	(dummy2)	--	(CHCO);
	\end{tikzpicture}
	}
	\caption{Schematische Darstellung der biotischen Gasherstellung \parencite{KGKK2019}; \textit{Eigene Darstellung}}
	\label{fig:shem_biogas-production}
	\end{center}
\end{figure}