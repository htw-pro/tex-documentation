\section{Biomethanpotential}

Das Biomethanpotential in Deutschland ist nur zu geringen Teilen erschlossen. Derzeit werden jährlich etwa \SIrange{96}{106}{\twhHs} Biogas in Deutschland erzeugt, wovon rund \SI{9}{\twhHs} zu Biomethan aufbereitet werden. Es liegen verschiedene Studien vor, die das zukünftige Biomethanpotential in Deutschland abschätzen. In diesem Kapitel soll ein zusammenfassender Überblick über die wichtigsten Kennzahlen des Biomethanpotentials in Deutschland geschaffen werden. \smallskip

So ist bereits das Potential für die Erzeugung von Biogas aus kommunalen und industriellen Abfällen und Reststoffen, tierischen Exkrementen, Energiepflanzen und Stroh ist bis heute bei weitem nicht ausgeschöpft. Beispielsweise wurden \SI{2016}{\relax} nur \SI[per-mode=symbol]{89000}{\tonne\peranno} der \SI[per-mode=symbol]{4446000}{\tonne\peranno} angefallenen Bioabfällen energetisch verwertet. Weiterhin kann die Erzeugung durch Repowering und Effizienzsteigerungen von Bestandsanlagen weiter erhöht werden. Im Schnitt weisen alle betrachteten Studien ein Potential von etwa \SI{100}{\twhHs} Biomethan bis zum Jahr 2030 aus. In Tab. \ref{tab:tab_methan-potential} findet sich ein Überblick über die Ergebnisse der betrachteten Studien. \parencite{BDEW2019a} \parencite{dena2017} \parencite{WIKUE2006}

{
\renewcommand{\arraystretch}{1.1}
\begin{table}[H]
	\begin{center}
		\caption{Biomethanpotential in Deutschland}
		\begin{tabu} to 0.49\textwidth {X X[1.2, r]}
			\hline
			Quelle             &	Biomethanpotential in \si{\twh}																\\ \hline
			BDEW \parencite{BDEW2019a}              	& 	\SI{100}{\relax}$^{\mathrm{a}}$ bis \SI{250}{\relax}$^{\mathrm{b}}$	\\
			dena \parencite{dena2017}              		& 	\SIrange{90}{118}{\relax}											\\
			Wuppertal Institut \parencite{WIKUE2006}	& 	\SIrange{78}{105}{\relax}$^{\mathrm{a}}$							\\ \hline
			\multicolumn{2}{l}{$^{\mathrm{a}}$2030 $^{\mathrm{b}}$2050 } 														\\
		\end{tabu}
		\label{tab:tab_methan-potential}
	\end{center}
\end{table}
}