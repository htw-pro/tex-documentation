\section{Einleitung}

Der Ausbau an \glspl{FEE} führt zu einem erhöhten Bedarf an Flexibilitätsoptionen, um einen Ausgleich zwischen Angebot und Nachfrage zu erzielen. So kommt es in einem zukünftigen Energiesystem, welches von \glspl{FEE} dominiert wird, zu einer starken Verschiebung der Residuallastkurve. Als steuerbare regenerative Erzeugungseinheiten, können Biogasanlagen eine wichtige Rolle in der Erbringung von Flexibilität übernehmen und zu einem abfedern der Residuallast beitragen. \parencite{ISE2013}\smallskip

Die Umstellung der Fahrweise von Biogasanlagen, weg von einer Maximierung der Volllaststundenzahl hin zu einer flexiblen Erzeugung, bedeutet einen erhöhten planerischen, technischen und operativen Aufwand und somit erhöhte Kosten gegenüber dem Status quo. Um einen Anreiz hin zu einer Flexibilisierung bestehender Anlagenleistung zu schaffen, wurde mit dem \gls{EEG} 2012 (\S 33i) die Flexibilitätsprämie eingeführt. Anschließend wurde mit dem \gls{EEG} 2014 (\S 53) das Anreizprogramm durch die Einführung des Flexibilitätszuschlages auf Neuanlagen ausgeweitet und mit dem \gls{EEG} 2017 auf ein Ausschreibunssystem umgestellt. \parencite{DanielGromke2019}

% ToDo: Hier muss noch Gasaufbereitung zu Biomethan als Möglichkeit genannt werden.
% ToDo: Warum Biomethan? und nicht einfach Biogas?
% ToDo: Was ist das Ziel dieser Arbeit? beantworten
	% e.g. aufzeigen, ob Gasaufbereitung zu Biomethan genügend Potential bietet, Erlösströme zu generieren
% ToDo: Konzentrieren auf Flexibilitätsprämie, da Titel "Bestandssicherung"
% ToDo: Bestandanlagen definieren (Anlagen die vor dem 01.08.2014 ans Netz gingen)
% s. BDEW Gas kann grün:Die Potentiale vonBiogas/Biomethan https://www.bdew.de/media/documents/Awh_20190426_Gas-kann-gruen-Potentiale-Biogas.pdf