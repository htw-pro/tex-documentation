\section{Einleitung}

Der Ausbau an \glspl{FEE} führt zu einem erhöhten Bedarf an Flexibilitätsoptionen, um einen Ausgleich zwischen Angebot und Nachfrage zu erzielen. So kommt es in einem zukünftigen Energiesystem, welches von \glspl{FEE} dominiert wird, zu einer starken Verschiebung der Residuallastkurve. Als steuerbare regenerative Erzeugungseinheiten, können Biogasanlagen eine wichtige Rolle in der Erbringung von Flexibilität übernehmen und zu einem abfedern der Residuallast beitragen. \parencite{ISE2013}\smallskip

Eine Möglichkeit Biogasbestandsanlagen zu flexibilisieren, stellt die Aufbereitung von Biogas zu Biomethan dar, welches anschließend in das Gasnetz eingespeist werden kann. Die Umstellung der Fahrweise von Biogas- bzw. Biomethananlagen, weg von einer Maximierung der Volllaststundenzahl hin zu einer flexiblen Erzeugung, bedeutet einen erhöhten planerischen, technischen und operativen Aufwand und somit erhöhte Kosten gegenüber dem Status quo. Um einen Anreiz hin zu einer Flexibilisierung bestehender Anlagenleistung zu schaffen, wurde mit dem \gls{EEG} 2012 (\S 33i) die Flexibilitätsprämie eingeführt. Anschließend wurde mit dem \gls{EEG} 2014 (\S 53) das Anreizprogramm durch die Einführung des Flexibilitätszuschlages für Neuanlagen ausgeweitet. \parencite{DanielGromke2019}\smallskip

Ziel dieser Arbeit ist es, die Biogasaufbereitung zu Biomethan als Option zur Flexibilisierung in dem aktuellen rechtlichen Umfeld aus technischer, wirtschaftlicher und ökologischer Perspektive zu bewerten.