% Shortcuts for SI Values
% Grundlagen: https://www.namsu.de/Extra/pakete/Siunitx.html
% Gute Erklärung der Shortcuts: https://texwelt.de/fragen/2588/wie-schreibe-ich-zahlen-mit-einheiten-richtig

%\DeclareSIUnit{\BeladungsDichte}{\kilo\gram_{\textup{H\textup{2}}\per\kilo\gram_{\textup{FeTi}}}}		% Beladungsdichte
%\DeclareDocumentCommand\BeladungsDichte{O{}m}{\SI[#1]{#2}{\BeladungsDichte}}

%\DeclareSIUnit[]\NormVolumen
%{\text{\ensuremath{\cubic\meter_{\textup{i.N.}}}}}

%%%%%%% New SIValues

\DeclareSIUnit{\sieuro}{\mbox{\euro{}}}
\DeclareSIUnit\ct{ct}
\DeclareSIUnit\normvol{\cubic\meter_{i.N.}}
\DeclareSIUnit\normvolL{\liter_{i.N.}}
\DeclareSIUnit\kw{\kilo\watt}
\DeclareSIUnit\mw{\mega\watt}
\DeclareSIUnit\gw{\giga\watt}
\DeclareSIUnit\kwh{\kilo\watt\hour}
\DeclareSIUnit\gwh{\giga\watt\hour}
\DeclareSIUnit\twh{\tera\watt\hour}
\DeclareSIUnit\kwp{\kilo\watt_{p}}
\DeclareSIUnit\ctkwh{\ct\per\kwh}
\DeclareSIUnit\Eurkwh{\sieuro\per\kwh}
\DeclareSIUnit\Eurkw{\sieuro\per\kw}
\DeclareSIUnit\Jahre{Jahre}
\DeclareSIUnit\Jahren{Jahren}

%%%%%%% New complete Commands

\NewDocumentCommand\DeclareNewQuantity{mmm}{%
	\DeclareSIUnit{#2}{#3}%
	\DeclareDocumentCommand{#1}{O{}m}{\SI[##1]{##2}{#2}}%
}

\DeclareNewQuantity
	\Dichte
	\dichte
	{\kg\per\cubic\meter}
\DeclareNewQuantity
	\Normvolumen
	\normvolumen
	{\normvol}
