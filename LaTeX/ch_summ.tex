% Summary

\section{Zusammenfassung}\label{ch_sum}

Der erste Schritt zur Herstellung von Biomethan aus Biogas ist das Abtrennen des enthaltenen \ce{CO2}. Dabei ist es das Hauptziel eine Methanreinheit von \SI{95}{\percent} oder mehr zu erreichen, damit das Biomethan ins Gasnetz eingespeist werden darf. Aktuell gibt es sechs verschiedene Verfahren die dafür in Deutschland zum eingesetzt werden, die Druckwasserwäsche \gls{DWW}, Aminwäsche, organische Wäsche, Druckwechseladsorption \gls{PSA}, Membrantrennverfahren und kryogene Trennverfahren. Es wird angenommen, dass die \gls{DWW} für einzelne Biogasbestandsanlagen das am Besten geeignete Biogasaufbereitungsverfahren ist. Es ist vergleichsweise preiswert in seiner Anschaffung, gerade auch für kleine Biogasströme, und mit Aufbereitungskosten von \SIrange{1,2}{1,5}{\ct\per\kwh} bezogen auf den Brennwert von Methan recht preiswert. Außerdem ist Wasser als Lösungsmittel kostengünstig und ökologisch unbedenklich.  \newline
Für das abgeschiedene \ce{CO2} besteht bisher kein relevanter Absatzmarkt. Je nach Entwicklung von PtG-Techologien kann das jedoch eine Option für die Zukunft werden. Nach der Gasaufbereitung kann das \ce{CO2} stattdessen mithilfe von \ce{H2} zu Methan synthetisiert werden, um die Biomethanproduktion zu erhöhen. Zum gegenwärtigen Zeitpunkt ist das allerdings nicht zu empfehlen, da die nötigen Investitionskosten, nicht nur für die Methanisierungsanlage sondern auch für den Elektrolyseur, sehr hoch sind im Vergleich zum Nutzen der daraus gezogen werden kann. \newline
Der Preis für Biomethan beläuft sich insgesamt auf ca. \SIrange{6,2}{8,0}{\ct\per\kwh}. Eine Umstellung von reinen Biogasanlagen zu Biomethaneinspeisung ist verbunden mit hohen Investitionskosten und einem vergleichsweise unsicheren Absatzmarkt. Anlagen mit hohen Biogasströmen sind dabei bevorzugt, da die spezifischen Kosten mit steigender Kapazität deutlich fallen. Ein wirtschaftlicher Betrieb ist derzeit dennoch möglich, vor allem durch Förderung, wie beispielsweise die Zahlung vermiedener Netzentgelte. Um ein langfristig wirtschaftliches Geschäftsmodell zu etablieren, ist es trotzdessen dringend notwendig, dass weitere Anreize und rechtliche Rahmenbedingungen geschaffen werden, die den Stand von Biomethan, insbesondere im Vergleich zu fossilem Erdgas, verbessert.
