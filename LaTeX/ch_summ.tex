% Summary

\section{Schlussbetrachtung}\label{ch_sum}

In Deutschland besteht ein großes Potential zur Erzeugung von Biomethan. Werden die richtigen regulatorischen Rahmenbedingungen gesetzt, ist eine jährliche Erzeugung von ca. \SI{100}{\twh} an Biomethan realistisch. Doch derzeit besteht aus rechtlicher Sicht ein eher unattraktives Umfeld für die Flexibilisierung von Biogasbestandsanlagen durch die Herstellung von Biomethan. Insgesamt wurden durch das \glsfirst{EEG} zwar Anreize für eine Flexibilisierung des Anlagenparks geschaffen, jedoch sind aufgrund der Rahmenbedingungen alternative Flexibilisierungsoptionen attraktiver. Dies wird ausgelöst, durch die Begrenzung der Förderdauer der vermiedenen Netzentgelte auf zehn Jahre nach der Gasnetzentgeltverordnung, und dem höheren Korrekturfaktor von Biomethan gegenüber Biogas im \gls{EEG}. Für Anlagen, die aus dem \gls{EEG} fallen, wurde bisher auf regulatorischer Ebene kein Anreiz zur Flexibilisierung geschaffen.\smallskip

Der erste Schritt zur Herstellung von Biomethan aus Biogas ist die Abtrennung des enthaltenen Kohlenstoffdioxids. Dabei ist es das Hauptziel eine Methanreinheit von mindestens \SI{95}{\percent} zu erreichen, damit das Biomethan ins Gasnetz eingespeist werden darf. Aktuell gibt es sechs verschiedene Verfahren die dafür in Deutschland eingesetzt werden, die \glsfirst{DWW}, Aminwäsche, organische Wäsche, Druckwechseladsorption \gls{PSA}, das Membran- und kryogene Trennverfahren. Derzeit stellt die \gls{DWW} für Biogasbestandsanlagen das am Besten geeignete Biogasaufbereitungsverfahren dar. Es ist vergleichsweise preiswert in seiner Anschaffung, gerade auch für kleine Biogasströme, und mit Aufbereitungskosten von \SIrange{1,2}{1,5}{\ct\per\kwh} bezogen auf den Brennwert von Methan recht günstig im Betrieb. Außerdem ist Wasser als Lösungsmittel kostengünstig und ökologisch unbedenklich.\smallskip

Der Preis für Biomethan beläuft sich insgesamt auf ca. \SIrange{6,2}{8,0}{\ct\per\kwh}. Eine Umstellung von reinen Biogasanlagen zu Biomethaneinspeisung ist verbunden mit hohen Investitionskosten und einem vergleichsweise unsicheren Absatzmarkt. Anlagen mit hohen Biogasströmen sind dabei bevorzugt, da die spezifischen Kosten mit steigender Kapazität deutlich fallen. In bestimmten Fällen ist ein wirtschaftlicher Betrieb durch Förderungen und die zehnjährige Zahlung vermiedener Netzentgelte derzeit dennoch möglich. Um ein langfristig wirtschaftliches Geschäftsmodell zu etablieren, ist es dringend notwendig, dass weitere Anreize und rechtliche Rahmenbedingungen geschaffen werden, die den Stand von Biomethan, insbesondere im Vergleich zu fossilem Erdgas, verbessern. \smallskip

Für das abgeschiedene Kohlenstoffdioxid besteht bisher kein relevanter Absatzmarkt. Nach der Gasaufbereitung kann das abgeschiedene Kohlenstoffdioxid zusammen mit Wasserstoff zu Methan synthetisiert werden. Je nach der Entwicklung der \gls{PtG}-Techologien kann dies jedoch zukünftig eine Vermarktungsoption darstellen. Zum gegenwärtigen Zeitpunkt ist dies allerdings aus ökonomischer Sicht nicht zu empfehlen, da die nötigen Investitionskosten, nicht nur für die Methanisierungsanlage sondern auch für den Elektrolyseur, sehr hoch sind im Vergleich zum Nutzen der daraus gezogen werden kann.\smallskip

Aus ökologischer Sicht wäre die Nutzung des abgeschiedenen Kohlenstoffdioxids ein großer Erfolg. Doch auch ohne die Nutzung des abgeschiedenen Kohlenstoffdioxids, können Biomethananlagen eine durchaus positive Treibhausgasbilanz aufweisen. Hierfür ist eine möglichst hohe Effizienz der Anlage, die Vermeidung von Leckagen und die Wahl des Ausgangssubstrat entscheidend.
