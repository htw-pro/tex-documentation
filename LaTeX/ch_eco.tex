\section{Ökologische Bilanz}

Der ökologische Fußabdruck von Biogas und -methan ist stark abhängig von den Prozessparametern und dem verwendeten Substrat. Bei einer Betrachtung der ökologischen Auswirkungen müssen neben den Treibhausgasemissionen weitere Aspekte beachtet werden. Daher sollen folgende Aspekte in diesem Kapitel kurz diskutiert und dargestellt werden:

\begin{itemize}
	\item Treibhausgasemissionen
	\begin{itemize}
		\item \ce{CO2}-Emissionen
		\item \ce{CH4}-Emissionen
	\end{itemize}
	\item Luftschadstoffe
	\item Lagerung und Behandlung des Gärguts
	\item Endnutzung des Gärguts
	\item Feinstaub-Emissionen
	\item Aufbereitung zu Biomethan
\end{itemize}


\subsection{Treibhausgasemissionen}

Bei dem Gärprozess von Biogas kommt es zur Produktion verschiedener Treibhausgase. Die wichtigsten Treibhausgase sind hierbei Kohlenstoffdioxid (\ce{CO2}) und Methan (\ce{CH4}). Die Reduktion der Emissionen dieser Treibhausgase stellt eine der wichtigsten Aufgaben der Biogas- und Biomethanproduktion dar, damit fossile Energieträger effektiv ersetzt werden können. Die wesentlichsten Maßnahmen zur Reduktion der Treibhausgasemissionen sind eine Gasfackel um \ce{CH4}-Emissionen zu vermeiden, ein dichter Tank, ein hoher Wirkungsgrad des Blockheizkraftwerks, eine konsequente Nutzung der Wärmeenergie und die Vermeidung von Leckagen. \parencite{Paolini2018}


\subsubsection{Kohlenstoffdioxid-Emissionen}

\ce{CO2}-Emissionen in der Biogas- bzw. Biomethankette entstehen in erster Linie während der Verbrennung des Gases im Blockheizkraftwerk. Bei einer Biogaszusammensetzung von etwa \SI{65}{\percent} \ce{CH4} und \SI{35}{\percent} \ce{CO2} belaufen sich die \ce{CO2}-Emissionen der Verbrennung von Biogas auf etwa \SI[per-mode=symbol]{301}{\gcoeqkwh}. Andere Emissionsquellen stellen vor allem der Transport, die Lagerung (s. Kap. \ref{chap:digestate_storage}) und die Endnutzung des Gärguts (s. Kap. \ref{chap:digestate_use}) dar. \parencite{Nielsen2014} \parencite{Paolini2018} \smallskip

Demgegenüber steht, dass Biogas aus biogenen Stoffen gewonnen wird und somit das \ce{CO2} zuvor gebunden wurde. Durch die Verdrängung von fossilen Brennstoffen, kann somit eine positive \ce{CO2}-Bilanz erreicht werden. Dies ist jedoch stark davon abhängig, aus welchem Substrat das Biogas gewonnen wurde. Auf diesen Zusammenhang wird im Kap. \ref{chap:eco_biomethane} vertieft eingegangen.


\subsubsection{Methan-Emissionen}

Mit einem Treibhauspotential, welches über einen Zeitraum von \SI{100}{\Jahren} dem \SIrange{28}{36}{\relax}-fachen von \ce{CO2} entspricht, stellt \ce{CH4} den zweitgrößten Anteil am anthropogenen Treibhauseffekt dar. Zu \ce{CH4}-Emissionen kommt es durch unvollständige Verbrennung, Leckagen und diffuse Emissionen während der Lagerung und Behandlung (s. Kap. \ref{chap:digestate_storage}) und der Endnutzung des Gärguts (s. Kap. \ref{chap:digestate_use}). \parencite{Paolini2018}


\subsection{Luftschadstoffe}

Bei der Verbrennung von Biogas bzw. -methan, entstehen neben Treibhausgasen, auch Luftschadstoffe. Die wichtigsten Luftschadstoffe hierbei sind Kohlenstoffmonoxid (\ce{CO}), Schwefeldioxid (\ce{SO2}), Stickstoffoxide (\ce{NO_X}), flüchtige organische Verbindungen (\textit{non-methane volatile organic compounds} (\ce{NMVOC})) und Formaldehyd (\ce{CH2O}). In Tab. \ref{tab:tab_air-pollutants} findet sich eine Übersicht über übliche Emissionsgrade der direkten Verbrennung von Biogas.

{
\renewcommand{\arraystretch}{1.1}
\begin{table}[H]
	\begin{center}
		\caption{Emissionsfaktoren von Biogasanlagen mit direkter Biogasverbrennung \parencite{Paolini2018}}
		\begin{tabu} to 0.5\textwidth {| X | X[1.5] |}
			\hline
			Schadstoff	& Emissionen in \si[per-mode=symbol]{\mgkwh}					\\ \hline
			\ce{CO}		& \SIrange{922}{1116}{\relax}                               	\\
			\ce{SO2}	& \SI{90}{\relax}                                       		\\
			\ce{NO_X}	& \SIrange{727}{1944}{\relax}                               	\\
			\ce{NMVOC}	& \SIrange{36}{76}{\relax}                                  	\\
			\ce{CH2O}	& \SIrange{31}{50}{\relax}                                  	\\ \hline
		\end{tabu}
		\label{tab:tab_air-pollutants}
	\end{center}
\end{table}
}

Auch bei den Luftschadstoffen gibt es eine starke Abhängigkeit der Emissionsgrade von dem gewählten Substrat. Eine möglichst geringe Belastung mit Luftschadstoffen ist kritisch für die Akzeptanz in der Bevölkerung. \parencite{Paolini2018}

\subsection{Lagerung und Behandlung des Gärguts}\label{chap:digestate_storage}

Die richtige Lagerung und Behandlung des Gärguts stellt einen der wichtigsten Punkte zur Reduzierung der Treibhausgas- und Ammoniak-Emissionen dar. Ein gasdichter Tank ist Voraussetzung für eine möglichst ökologische Bereitstellung von Biogas bzw. -methan. \parencite{Paolini2018}


\subsection{Endnutzung des Gärrests}\label{chap:digestate_use}

Der wichtigste ökologische Faktor der Endnutzung des Gärguts ist der Nitrateintrag in die Umwelt. Es muss zwingend darauf geachtet werden, dass ein angebrachtes Verteilungsmanagement des Gärrests zur Anwendung kommt, damit die Boden- und Wasserqualität nicht unnötig stark belastet wird.\smallskip

Zusätzlich kann es durch die Ausbringung von unbehandeltem Gärresten zu starken Emissionen von Methan, Distickstoffmonoxid, Ammoniak, flüchtigen Kohlenwasserstoffe und anderen Chemikalien kommen. Durch geeignete Behandlungsmethoden kann dieses Potential an Treibhausgas- und anderen Emissionen deutlich gesenkt werden. \parencite{Paolini2018}


\subsection{Aufbereitung zu Biomethan}\label{chap:eco_biomethane}

Da Biogas das Vorprodukt von Biomethan darstellt, sind alle zuvor beschriebenen ökologischen Auswirkungen ebenfalls Biomethan anzurechnen. Hiervon ausgenommen sind die verbrennungsabhängigen Emissionen des Biogases, welche für Biomethan aufgrund der Zusammensetzung mit denen von Erdgas vergleichbar sind.\smallskip

Die Umweltverträglichkeit von Biomethan hängt maßgeblich von der \ce{CH4}-Leckagerate in der Prozesskette ab. So bietet Biomethan bei einer Leckagerate von \SI{4}{\percent} bei der Betrachtung der Treibhausgasemissionen keine Vorteil mehr gegenüber der fossilen Erzeugung. \SI{2011}{\relax} wurde die Leckagerate einer Anlage zur Biomethanherstellung und -verbrennung der E.ON Ruhrgas AG mit einem Wert von \SI{0.1}{\percent} bestimmt. Die Gesamtbilanz beläuft sich trotz der Nutzung von Energiepflanzen  auf lediglich \SI[per-mode=symbol]{44.6}{\gcoeqkwh} welches einer Emissionsreduktion von \SI{82}{\percent} gegenüber einer Erzeugung mit Erdgas entspricht. \parencite{Adelt2011} \parencite{Ravina2015} \smallskip

Die Treibhausgasemissionen der Prozesskette hängen dabei stark von dem Ausgangssubstrat ab. So kann die Herstellung von Biomethan vor allem bei der Verwendung von Reststoffen eine negative und bei Energiepflanzen eher eine positive Treibhausgasbilanz aufweisen. So können die Treibhausgasemissionen je nach Ausgangssubstrat zwischen etwa \SIrange{-800}{210}{\gcoeqmj} schwanken. Die beste Treibhausbilanz weist hierbei Rindergülle auf, während Molke die schlechteste Bilanz aufweist. \parencite{Tonini2016}


\subsection{Fazit}

Zusammenfassend lässt sich sagen, dass die ökologischen Auswirkungen der Biomethanbereitstellung von Anlage zu Anlage stark unterschiedlich sind. Unter den richtigen Rahmenbedingungen bietet Biomethan jedoch eine durchaus nachhaltige Alternative zu Erdgas.\smallskip

Um eine Minimierung an Treibhausgasen, Luftschadstoffen und Nitrateintrag in die Umwelt zu erreichen, sollten den folgenden Eckpunkte besondere Beachtung geschenkt werden:

\begin{itemize}
	\item Eine effiziente und möglichst vollständige Verbrennung
	\item Eine möglichste vollständige Nutzung der Wärme
	\item Die Vermeidung von Leckagen
	\item Eine geeignete Behandlung des Gärrests
	\item Der Wahl des Ausgangssubstrats
\end{itemize}