\section{Table  Example}

\begin{table}[htbp]
	\begin{center}
		\begin{tabular}{|c|c|c|c|}
			\hline
			\textbf{Table}&\multicolumn{3}{|c|}{\textbf{Table Column Head}} \\
			\cline{2-4} 
			\textbf{Head} & \textbf{\textit{Table column subhead}}& \textbf{\textit{Subhead}}& \textbf{\textit{Subhead}} \\
			\hline
			copy& More table copy$^{\mathrm{a}}$& &  \\
			\hline
			\multicolumn{4}{l}{$^{\mathrm{a}}$Sample of a Table footnote.}
		\end{tabular}
		\caption{Table Type Styles}
		\label{tab1}
	\end{center}
\end{table}

\section{Zwei}

\subsection{Figure Example}

%\begin{figure}[htbp]
%	\centerline{\includegraphics{fig1.png}}
%	\caption{Example of a figure caption.}
%	\label{fig}
%\end{figure}

\subsection{Acronyme Example}

Given a set of numbers, there are elementary methods to compute its \acrlong{gcd}, which is abbreviated \acrshort{gcd}. \parencite[s.][]{2020_latex}

\noindent \subsubsection{ZweiPunktZweiPunktEins}

\noindent \subsubsection{ZweiPunktZweiPunktZwei}