% abstract

\section*{Abstract}

\forceindent In dieser Arbeit werden die gängigsten Biogasaufbereitungstechnologien vorgestellt und technisch sowie ökonomisch verglichen. Außerdem werden mögliche Verfahren für die direkte und indirekte Methanisierung des im Biogas enthaltenen Kohlenstoffdioxids aufgezeigt und der derzeitige rechtliche Rahmen, sowie die ökologische Bilanz des Biomethans bewertet. \smallskip

\forceindent Die \gls{DWW} wird als günstigstes Verfahren zur Biomethanherstellung herausgearbeitet. Die Kosten der Aufbereitung betragen etwa \SIrange{1,2}{1,5}{\ct\per\kwh} Methan. Für die gesamte Prozesskette werden Kosten von \SIrange{6,2}{8,0}{\ct\per\kwh} erwartet. \smallskip

\forceindent Grundsätzlich kann die Umgestaltung Biogasanlagen zur Biomethaneinspeisung wirtschaftlich sein. Allerdings sind die meisten bestehenden Biogasanlagen abhängig von zeitlich begrenzten Fördermitteln. Aufgrund der angespannten regulatorischen Lage, ist bereits seit \SI{2017}{\relax} ein rückläufiger Trend beim Zubau von Biomethananlagen festzustellen. Werden die rechtlichen Rahmenbedingungen nicht angepasst, ist davon auszugehen, dass die installierte Anlagenleistung weiter sinkt. Dies gilt vor allem für Anlagen, die ab \SI{2020}{\relax} sukzessive ihre Vergütung nach dem \glspl{EEG} verlieren, aber auch für neue Anlagen nach Ablauf des Förderzeitraums. \smallskip

\forceindent Biomethananlagen können eine durchaus positive Treibhausgasbilanz aufweisen. Hierfür ist eine möglichst hohe Effizienz der Anlage, die Vermeidung von Leckagen und die Wahl des Ausgangssubstrat entscheidend.